\documentclass[landscape]{slides}
\usepackage{amsmath}
\usepackage{amssymb}
\usepackage{color}
\begin{document}

\begin{tabular}{r|p{0.7 \linewidth}}
	\textbf{Número:} &  3.5.6.\\ 
	\textbf{Ubicación:} &  Página 6/8.\\ 
	\textbf{Texto:} & Dado un índice
	\textcolor{red}{
	\[
		k \in I	
	\]
	}
	si, para cada índice 
		\[
		i \in I - \{j\}
	\]
	existe un \(\mathbf{C}-\)morfismo
	\[
		f_i:A_j \longrightarrow A_i
	\]
	(...)
	% hay que usar color ROJO para indicar la errata.
	\\ 
	\textbf{Corrección:} & 
	si, para cada índice \textcolor{blue}{\[
		j \in I
	\]}
	existe un \(\mathbf{C}-\)morfismo (...)
	% hay que usar color AZUL para indicar la errata.
\end{tabular}

\end{document}
