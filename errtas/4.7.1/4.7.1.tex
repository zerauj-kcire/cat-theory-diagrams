\documentclass[landscape]{slides}
\usepackage{amsmath}
\usepackage{amssymb}
\usepackage{color}
\begin{document}

\begin{tabular}{r|p{0.7 \linewidth}}
	\textbf{Número:} & 4.7.1.(x) \\ 
	\textbf{Ubicación:} & Página 2/5. \\ 
	\textbf{Texto:} & 
	(...) queda definida la aplicación 
	\[
		1_{A ^ *} : A ^ * \longrightarrow A ^ * 
	\]
	\[
		h \longmapsto \textcolor{red}{h \circ 1_A}
	\]
	que es una función identidad entre clases.
	% hay que usar color ROJO para indicar la errata.
	\\ 
	\textbf{Corrección:} & 
	(...) queda definida la aplicación 
	\[
		1_{A ^ *} : A ^ * \longrightarrow A ^ * 
	\]
	\[
		h \longmapsto \textcolor{blue}{1_A \circ h}
	\]
	% hay que usar color AZUL para indicar la errata.
\end{tabular}

\end{document}
