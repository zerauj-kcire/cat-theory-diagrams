\documentclass[landscape]{slides}
\usepackage{amsmath}
\usepackage{amssymb}
\usepackage{mathrsfs}
\usepackage{color}
\begin{document}
\begin{tabular}{r|p{0.7 \linewidth}}
	\textbf{Número:} &  1.1.4. \\ 
	\textbf{Ubicación:} & Pág. 4. Sección 1.1. \\ 
	\textbf{Texto:} & 
	\begin{enumerate}
		\item[\textbf{(Cat2)}] Si \(f,g,h \in \overrightarrow{\mathscr{C}}\),
			tales que
			\[
				cod(f) = dom(g) \;,\; cod(g) = dom(h),
			\]
			entonces \textcolor{red}{\(h \circ (g \circ f) = (h \circ g) \circ f\).}
		\item[\textcolor{red}{\textbf{(Cat3)}}] 
			Si \(f,g \in \overrightarrow{\mathscr{C}}\),
			son tales que \(cod(f) = dom(g)\), entonces
			\[
				dom(g \circ f) = dom(f).
			\]
		\item[\textcolor{red}{\textbf{(Cat4)}}] 
			Si \(f,g \in \overrightarrow{\mathscr{C}}\),
			son tales que \(cod(f) = dom(g)\), entonces
			\[
				cod(g \circ f) = cod(g).
			\]
	\end{enumerate}
	% hay que usar color ROJO para indicar la errata.
\end{tabular}
\newpage
\begin{tabular}{r|p{0.7 \linewidth}}
	\textbf{Explicación:} & Se requiere saber primero que si
	\[
		cod(g) = dom(h),
	\]
	entonces
	\[
		cod(g \circ f) = dom(h).
	\] 
	Para componer los \(3\) morfismos en \textbf{(Cat2)}.\\ 
	\textbf{Sugerencia:} & 
	\begin{enumerate}
		\item[\textcolor{blue}{\textbf{(Cat2)}}] 
			Si \(f,g \in \overrightarrow{\mathscr{C}}\),
			son tales que \(cod(f) = dom(g)\), entonces
			\textcolor{blue}{\(dom(g \circ f) = dom(f).\)}
		\item[\textcolor{blue}{\textbf{(Cat3)}}] 
			Si \(f,g \in \overrightarrow{\mathscr{C}}\),
			son tales que \(cod(f) = dom(g)\), entonces
			\textcolor{blue}{\(cod(g \circ f) = cod(g).\)}
		\item[\textbf{(Cat4)}] Si \(f,g,h \in \overrightarrow{\mathscr{C}}\),
			tales que
			\[
				cod(f) = dom(g) \;,\; cod(g) = dom(h),
			\]
			entonces \textcolor{blue}{\(h \circ (g \circ f) = (h \circ g) \circ f\).}
	\end{enumerate}
	% hay que usar color AZUL para indicar la errata.
\end{tabular}

\end{document}
